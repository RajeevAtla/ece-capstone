\documentclass[conference]{IEEEtran}
\IEEEoverridecommandlockouts
% The preceding line is only needed to identify funding in the first footnote. If that is unneeded, please comment it out.
\usepackage{cite}
\usepackage{amsmath,amssymb,amsfonts}
\usepackage{algorithmic}
\usepackage{graphicx}
\usepackage{textcomp}
\usepackage{xcolor}
\def\BibTeX{{\rm B\kern-.05em{\sc i\kern-.025em b}\kern-.08em
    T\kern-.1667em\lower.7ex\hbox{E}\kern-.125emX}}
\begin{document}

\title{Smart Parking Lot Assist System\\
{\footnotesize \textbf{Interim Report}}
}

\author{
\textbf{Rajeev Atla, Parshva Mehta, Aman Patel, Abhiram Vemuri} \\
Advisor: Kristin Dana\\
Team Number: SP25-02 
}

\maketitle

\section{Introduction}
As transport infrastructure grows increasingly complex, 
large universities face mounting challenges in meeting the parking needs of students and faculty, 
especially with the continued reliance on individual car commuting. 
This project proposes an integrated solution using security camera imaging, 
computer vision, 
and a simple mobile application to ease the process of finding parking. 
The solution consists of two key components. 
First, 
a custom-configured security camera will be installed to overlook campus parking lots,
capturing high-resolution images. 
These images are processed through a computer vision model, 
pre-trained to detect open parking spaces with high accuracy. 
The output from this model is then converted into a dynamic dataset identifying available parking spots in real-time. 
Second, 
this dataset is seamlessly integrated into a user-friendly mobile application focused on clarity and ease of use. 
The app features an adaptive interface with real-time maps, 
filters, 
and notifications, 
enabling users to quickly locate free parking spaces near their destinations and make more informed parking decisions. 
Past attempts to improve parking with automated valet systems have failed, 
often because of high costs, 
complicated logistics, 
and issues with reliability. 
In a lively campus setting, 
these valet systems could be more feasible: 
changing class schedules, 
various modes of transportation, 
and limited staff availability make it hard to provide reliable valet services. 
As a result, 
drivers waste precious time driving in circles in busy lots, 
leading to congestion, 
frustration, 
and adverse environmental effects. 
By combining advanced imaging technology with intuitive design, 
this project aims to enhance the daily commuting experience for students and faculty. 
Moreover, 
integrating this information into an accessible platform contributes to the broader goal of fostering sustainable and efficient urban mobility patterns, 
paving the way for data-driven transportation solutions on and beyond university campuses.

\section{Approach Methods}

\subsection{Dataset Prep}

A diverse and robust dataset will form the foundation of this project. 
The dataset will include images of parking lot images of various occupancy levels. 
Images will be sourced from public datasets and proprietary captures using our camera to ensure relevance to the application. 
The preprocessing phase will involve resizing, 
normalization, 
and augmentation to prepare the data for model training and ensure data quality and robustness. 
These processes help simulate real-world scenarios and address potential image quality variations caused by environmental factors. 
Annotation tools will meticulously label parking spaces as occupied or vacant. 
Specific attention will be given to special categories like handicapped spots, 
electric vehicle charging spaces, 
and reserved parking areas, 
ensuring the system’s utility across diverse user needs. 
These steps will ensure the datasets’ quality and reliability for training, 
setting the stage for robust model performance.

\subsection{ML Model}

[4] https://docs.ultralytics.com/models/yolo11/

The project will leverage the YOLOv11 computer vision algorithm [4] to address the task of parking space detection. 
Known for its accuracy and speed, 
YOLO is well-suited for real-time applications.
The model will undergo customization and fine-tuning to align with the unique challenges of parking detection, 
such as variations in car sizes, 
overlapping boundaries, 
and partial occlusions caused by objects or shadows. 
By implementing advanced image processing techniques, 
the system will accurately identify parking spaces and determine whether they are vacant or occupied, 
even under diverse weather conditions and potential image quality disruptions. 
Training will focus on optimizing detection accuracy while minimizing computational demands to ensure seamless integration into low-power devices like the Raspberry Pi. 
Techniques like quantization and pruning will reduce model complexity without compromising performance. 
Rigorous testing and validation will ensure the model’s robustness across different environments, 
meeting the necessary standards for real-time operation. 
As previously mentioned, 
the datasets gathered in the data acquisition phase will be used to evaluate and refine the machine learning algorithm, 
improving its overall functionality and accuracy.

\subsection{Hardware}
Selecting a suitable Raspberry Pi and camera module will be critical to the project’s success. 
We need hardware that maintains stability 
(no outages/connection issues) 
and quality so the model runs smoothly. 
We will have a mounting system to maintain minimal movement so the resulting images are consistent. 
The testing phase will evaluate multiple aspects of the system, 
including stability and quality, 
and model detection accuracy across diverse parking lot layouts. 
Extensive field testing will help refine the integration and optimize the system for practical deployment. 
Suppose the Raspberry Pi system cannot run the machine learning model. 
In that case, 
we will pivot to running the model on an external computer and sending images from the camera to the computer. 
In this case, 
the Raspberry Pi will be used to connect via Wi-Fi or ethernet.

\subsection{Software}
The software component will integrate advanced image processing, 
machine learning, 
and user-facing functionalities. 
Captured images will undergo preprocessing steps such as noise reduction, 
color correction, 
and enhancement to ensure the highest possible input quality for the trained model. 
The inference pipeline will leverage the trained model to identify and classify real-time parking spaces, 
offering high accuracy and speed. 
A robust data transmission protocol, 
such as Wi-Fi or cellular communication, 
will facilitate seamless information sharing with a central server or directly with a mobile application. 
The mobile application will serve as the primary interface for end-users, 
providing real-time parking availability information through interactive maps, 
search functionality, 
and real-time notifications. 
Empty parking spaces will be highlighted on the interactive map for a given parking lot so that the user will receive this information. 
Using the machine learning algorithm and the data gathered from the camera, 
we can feed this into the backend of the mobile app function and apply a front-end user interface that connects with this information. 
Emphasis will be placed on designing a user-friendly experience that accommodates various devices and user preferences, 
ensuring wide adoption and usability.

\section{Challenges}
Throughout the development of our project, 
we encountered several significant challenges related to regulation and the supply chain, 
which ultimately limited the performance and scope of our system. 
Initially, 
the project was designed to utilize an aerial camera perspective, 
similar to the dataset images 
(see Figure 4), 
in order to accurately capture parking lines and define bounding boxes around each space. 
However, 
due to regulatory constraints from Rutgers University and the State of New Jersey, 
we were not permitted to install cameras on elevated platforms or drones as originally planned. 
As a result, 
we had to revise our approach and instead placed a stationary camera on the 5th floor of the CoRE building. 
This new vantage point introduced several visibility issues, 
as many parking spots were blocked by vehicles already parked in front of them, 
reducing the clarity and reliability of our input images. 

In addition to these regulatory limitations, 
we also faced delays in receiving essential hardware components. 
The camera's power supply and SD card arrived late in the semester, 
preventing us from fully setting up and testing the system on-site. 
This significantly limited our ability to collect real-world data from the intended parking lot, 
which in turn negatively impacted the accuracy and effectiveness of our trained model. 
The power supply and SD card arriving late has significantly impacted the ability to continue building the system pipeline. 
The goal was to have the camera take a continuous video, 
house the YOLOv11 model on the Raspberry Pi itself for local computing, 
take images at regular intervals with bounding boxers overlayed, 
send images to a database, 
pull the information from these images, 
and update a website with this information. 
Our development timeline has been dramatically affected by these issues.

Particularly on the software end of the project, 
the delay in receiving the hardware components needed to execute our project has drastically changed the way our database is being updated. 
The output of our ML model program needs to be used to update the database, 
but due to not having access to the power supply and SD card to fully test the system, 
we had to develop a workaround method for simply getting real data from the parking lot. 
The workaround we resorted to was using our phone cameras, 
which resulted in some limitations compared to what was originally envisioned such as lower quality resolution, 
and inconsistent frame rates of mobile devices. 
This made it harder for the ML algorithm to detect vacant or occupied parking spaces, 
which can result in missed or incorrect detections affecting the output file of the ML algorithm after running it and using the updated CSV file to update the database. 
These inaccuracies propagated through the pipeline resulting in a few false positives, 
resulting in a few incorrect entries into the database. 

Another problem that was encountered was when trying to optimize the refresh rate for database stability and accuracy. 
Initially our system was trying to update the database every 5 to 10 seconds from reading the CSV file, 
but this had several implications such as the database locking, 
writing conflicts, 
and also occasional crashes when running the server for frontend requests. 
In order to solve this the refresh interval was increased to 30 seconds to avoid instability in the database system. 
This eased the load on the database, 
kept the access conflicts with the frontend requests to a minimal amount, 
and allowed the ML model more time to process images and get higher quality outputs closer to the real-time scenario. 
Overall, 
this more refined approach allowed for a more sustainable balance between the backend functionality of the system and the real-time data generated by the ML model.

\section{Results}

During the model training phase, 
a fine-tuning strategy was employed due to the absence of prior knowledge regarding the optimal number of training epochs. 
This approach involved training the model for a fixed number of epochs to obtain an initial set of weights. 
These weights were then used to initialize the model for subsequent training cycles, 
effectively serving as a foundation for further refinement. 
This iterative weight generation and retraining process was repeated across multiple rounds, 
allowing for gradual performance improvements while mitigating the risk of overfitting or underfitting.

We begin by analyzing the training and validation loss curves to evaluate the learning behavior of the model throughout the training process. 
These plots provide insight into the model’s convergence and potential overfitting or underfitting tendencies.

\subsection{Training Loss}

Figure 1 illustrates the progression of the training loss throughout the training process. 
During the early epochs, 
the training loss exhibits fluctuations as the optimizer begins to interact with the training data. 
These fluctuations indicate the optimization process adjusting to the model’s parameters and gradually improving the model's ability to minimize the loss. 
In the subsequent epochs, 
the loss shows a steady decrease, 
demonstrating that the model is learning from the data and the optimizer is refining its parameters effectively.

The large fluctuations occur near the start of training, 
indicating a high learning rate near the start. 
The loss decreases drastically during this time. 
As the training progresses, 
the loss curve begins to plateau after around 750-800 epochs, 
with the loss stabilizing at approximately 0.6. 
This plateau indicates that the model is nearing convergence, 
meaning that the learning process is reaching a point where improvements are marginal, 
and the model's parameters have largely stabilized. 
The slight oscillations in the plot, 
particularly the ''spikes,'' 
correspond to moments of fine-tuning, 
where the optimizer undergoes specific adjustments to the model's weights. 
These fluctuations are expected, 
as fine-tuning allows for more nuanced adjustments, 
enhancing the model’s performance while avoiding drastic shifts in the learned parameters.

The consistent rise and fall of the loss during these fine-tuning periods reflects the adaptive nature of the training process. 
Notably, 
each fine-tuning cycle results in the loss ultimately decreasing to a lower value than the previous cycle, 
demonstrating effective model refinement. 
The general trend of the training loss decreasing over time and ultimately plateauing reinforces the model's capacity to converge toward an optimal solution.

[Figure 1: Training Loss Curve]

Figure 1 Caption: 
Notice how the training loss generally decreases throughout the training and plateaus after 750-800 epochs at a value of 0.6. 
The ''spikes'' in the beginning and throughout the plot represent fine-tuning.

\subsection{Validation Loss}

Figure 2 shows the progression of the validation loss throughout the training process. 
Initially, 
the validation loss decreases steadily, 
mirroring the trend observed in the training loss. 
However, 
as training continues, 
the rate of improvement slows, 
and the loss begins to plateau around 750-800 epochs. 
This plateau suggests that the model is reaching a point of diminishing returns, 
where further training would yield only marginal improvements in validation performance. 
The model’s ability to generalize to unseen data stabilizes, 
indicating that it has learned the underlying patterns effectively without overfitting.

The presence of spikes in the validation loss plot is similar to those in the training loss. 
These fluctuations correspond to periods of fine-tuning, 
where the optimizer makes targeted adjustments to the model’s weights. 
Fine-tuning helps refine the model's performance, 
ensuring it is better equipped to generalize to unseen data. 
These minor increases in validation loss reflect the optimizer's effort to fine-tune the learned parameters, 
balancing between exploration and exploitation of the parameter space. 
While these fluctuations are visible, 
they do not significantly detract from the overall downward trend of the validation loss.

The plateau observed after 750-800 epochs indicates that the model has achieved stability in its learning. 
At this stage, 
continuing training would likely lead to overfitting, 
where the model’s performance on the validation set would begin to degrade. 
Therefore, 
the training was halted at this point to avoid overfitting and ensure that the model maintained its generalization capability. 
The validation loss plot highlights the effectiveness of this strategy, 
demonstrating that the model achieved its optimal performance without unnecessary overtraining.

[Figure 2: Validation Loss Curve]

Figure 2 Caption: 
Notice how the validation loss generally decreases and plateaus after 750-800 epochs. 
When training the model, 
the stopping point was when the validation loss plateaus. 
The spikes are present in this plot as well.

\subsection{Evaluation}

[5] https://docs.ultralytics.com/guides/yolo-performance-metrics/

Figure 3 presents the evolution of the four evaluation metrics: 
Precision (P), 
Recall (R), 
mAP50, 
and mAP50-95. 
As shown, 
the values of all four metrics consistently increase over the course of training, 
indicating improvements in the model’s performance as it learns from more data. 
Precision steadily rises, 
reaching a final value of approximately 0.87, 
reflecting the model's growing accuracy in detecting objects. 
Recall increases as well, 
reaching around 0.80, 
suggesting that the model is becoming better at identifying all instances of objects in the images. 
Both mean average precision (mAP)
metrics demonstrate a similar upward trend, 
with mAP50 finishing at about 0.90 and mAP50-95 concluding at 0.72. 
These scores indicate improvements in the model's accuracy across different levels of detection difficulty and at various IoU thresholds.

The occasional spikes in the evaluation metrics are consistent with the fine-tuning process, 
as observed in the training and validation loss plots. 
These fluctuations represent moments when the optimizer adjusts the model’s weights, 
leading to short-term increases in performance before stabilizing again. 
Despite these minor oscillations, 
the overall trajectory of the metrics is clear: 
the model becomes more accurate and better able to generalize over time, 
as evidenced by the steady improvement across all four metrics.

These results are crucial for evaluating the model’s ability to perform object detection effectively. 
The improvements in precision and recall suggest that the model is not only becoming more accurate in its detections but is also identifying a larger proportion of the objects in the images. 
The increasing mAP scores further reinforce this, 
indicating that the model is refining its detection capabilities and performing well across various detection scenarios.

[Figure 3: Evaluation Metrics]

Figure 3 Caption:
The values of the four evaluation metrics 
(Precision, Recall, mAP50, and mAP50-95) 
improve steadily throughout the training process, 
indicating that the model's performance enhances as training progresses. 
The spikes in the plot reflect periods of fine-tuning and adjustments to the model’s weights.


\subsection{Test Examples}

In Figure 4, 
a test image from the dataset is presented with bounding boxes overlaid to indicate detected objects. 
Empty parking spaces are highlighted with green bounding boxes, 
while occupied spaces are marked with red. 
The YOLOv11 model successfully identifies and classifies parking spaces in this test image, 
demonstrating the model's ability to classify spaces in relative proximity.

However, 
as seen in Figure 5, 
the model encounters challenges when objects are further away, 
particularly with identifying vehicles and parking spaces at a distance. 
This limitation arises from the model's struggle to accurately detect and classify objects farther from the camera, 
likely due to the reduced resolution of distant objects and the difficulty distinguishing fine details at greater distances. 
The frame shown in Figure 5 is from a test video, 
where this issue impacts the model’s performance.

The difficulty in detecting distant objects suggests that parking lots with large or irregular layouts will require multiple cameras to ensure complete coverage. 
By placing cameras strategically throughout the lot, 
it will be possible to mitigate the model's struggles with distant detections, 
ensuring that all parking spaces, 
both near and far, 
are adequately monitored and classified. 
This approach will improve the system's overall accuracy and ensure more reliable performance across a broader area.

[Figure 4: Parking Lot Pic 1]
Figure 4 Caption: 
Observe that the empty parking spaces are marked with green bounding boxes while occupied spaces are marked with red bounding boxes. 
The YOLOv11 model correctly identifies and classifies spaces in the test data.

[Figure 5: Parking Lot Pic 2]
Figure 5 Caption: Observe that the model struggles to identify vehicles and spaces at a distance. 
Most lots will need more than one camera in order to cover every space. 
This is a frame taken from a test video.

\section{Cost/Sustainability analysis}

This project involves the development of a smart parking space detection system using a Raspberry Pi 5 (8GB) and a tilt-enabled camera setup. 
The system identifies empty and occupied parking spots to optimize parking efficiency and reduce time spent by drivers searching for available spaces. 
The prototype cost is approximately \$196.66, 
which includes the Raspberry Pi (\$80), 
tilt camera module (\$86.71), 
power supply (\$12), 
bumper (\$3), 
and a 64GB SD card (\$14.95). 
The estimated cost for mass production is approximately \$200 per unit, 
reflecting a cost-effective solution even at scale, 
especially considering the modularity and wide availability of its components.

Economically, 
the project is designed with affordability and scalability in mind. 
If deployed across large parking structures or citywide networks, 
further cost optimizations could be achieved through bulk procurement or centralized processing to reduce the need for one Raspberry Pi per camera. 
The system’s energy-efficient design also contributes to lower long-term operational costs, 
and potential tax incentives for smart city infrastructure, 
carbon footprint reduction, 
and energy-efficient technologies may reduce implementation expenses. 
Additionally, 
the system may provide indirect cost savings by lowering vehicle idling time, 
reducing fuel consumption, 
and minimizing emissions due to inefficient parking behaviors.

Environmentally, 
the system contributes positively by reducing greenhouse gas emissions associated with drivers circling to find parking. 
The hardware components, 
such as the Raspberry Pi, 
have a low power footprint and can operate efficiently with minimal energy consumption. 
Furthermore, 
the use of commonly available and recyclable materials supports a sustainable product lifecycle. 
As the system minimizes unnecessary vehicle movement, 
it indirectly reduces urban traffic congestion and resource overuse.

Socially, 
the impact of the project is notable in its ability to improve urban mobility, 
reduce driver frustration, 
and promote smarter city planning. 
By streamlining the parking process, 
it increases accessibility for all, 
including the elderly and those with disabilities, 
and reduces stress in high-traffic zones. 
While partial automation may reduce the need for manual lot attendants, 
it simultaneously opens avenues for new tech-related roles in installation, 
maintenance, 
and data analysis.
The project also enhances public safety by reducing the chances of illegal or hazardous parking and complies with evolving regulations around smart infrastructure and urban sustainability.

\section{Computer Vision/Society}

[1] https://ieeexplore.ieee.org/document/10009811

[2] https://ieeexplore.ieee.org/document/10914966

[3] https://ieeexplore.ieee.org/document/10332925

The spread of machine vision technologies has revolutionized urban infrastructure, 
positioning robotics at the forefront of societal innovation. 
These vision-based systems are integral to smart cities, 
driving automation in critical areas such as traffic management, 
security, 
and public services. 
By enabling real-time visual interpretation, 
these systems empower autonomous agents that improve urban life through enhanced efficiency, 
sustainability, and safety.

Recent advancements have shown that efficient, 
low-resource methods can now achieve vision-based object detection and tracking. 
An object-tracking approach utilizing frame-differencing, 
adaptive thresholding, 
and tracking algorithms, 
which enables real-time detection of moving objects with minimal computational overhead, 
was demonstrated [1]. 
Such techniques offer an affordable alternative to traditional hardware-intensive solutions, 
making intelligent automation more accessible to a broader range of applications. 
Concurrently, 
deep learning-based object detection systems, 
such as YOLO, 
have significantly advanced the capabilities of machine vision by balancing inference accuracy with computational efficiency, 
allowing real-time detection in everyday environments.

In the context of urban mobility, 
these advancements have been especially impactful in the development of smart parking systems. 
The work in [2] illustrates how machine vision can optimize parking space identification and classification, 
alleviating traffic congestion and maximizing space utilization in urban centers. 
Such systems can provide immediate, 
data-driven insights, 
improving user experience and urban planning. 
Automating routine tasks like parking identification reduces reliance on human labor and contributes to better traffic flow, 
environmental sustainability, 
and a more responsive urban environment.

Nevertheless, 
the widespread deployment of machine vision systems raises critical ethical concerns. 
Vision-based surveillance technologies, 
particularly in public spaces, 
present privacy, 
consent, 
and data security challenges. 
Unlike traditional surveillance systems, 
which often rely on biometric data, 
modern implementations like YOLO-based systems can focus solely on objects rather than individuals, 
offering a potential solution to privacy concerns. 
Despite this, 
future deployments must ensure privacy by design and transparency in how data is processed and used to maintain public trust and accountability.

Beyond urban mobility, 
integrating robotics into broader societal functions reflects a paradigm shift in public services. 
As [3] highlights, 
advances in social robotics show how robotic systems can enhance educational and medical services, 
particularly in pediatric and healthcare settings. 
Robots can assist in personalized learning, 
therapeutic activities, 
and medical monitoring by applying similar vision-based technologies. 
These advancements highlight the growing potential of robotics to serve not just as tools for automation but as agents of positive social change, 
addressing gaps in critical areas like education and healthcare while reinforcing the need for ethical considerations in their design and deployment.

The integration of robotic perception into civic infrastructure signals a shift in the relationship between technology and society. 
These systems highlight the growing role of automation in providing public services, 
raising important questions about responsibility, 
inclusivity, 
and governance in their design and deployment. 
As machine vision and robotics continue to evolve, 
their potential to reshape urban environments will depend on technical capabilities and their alignment with societal values and priorities.

In this evolving landscape, 
even low-cost systems like YOLO-based parking detection provide valuable insights into how vision-enabled robotics can be deployed for the public good. 
By examining their practical limitations and benefits, 
such systems offer important lessons in integrating intelligent technologies into public infrastructure while addressing ethical concerns and improving urban living conditions.

\section{Conclusion}

The project envisions a highly scalable and adaptable system capable of being deployed in a wide range of environments, 
including shopping malls, 
stadiums, 
airports, 
university campuses, 
and densely populated urban centers. 
Its modular design ensures seamless integration with existing infrastructure, 
making it suitable for both small-scale and large-scale parking facilities. 
By leveraging the portability and affordability of edge devices such as Raspberry Pi, 
the system remains cost-effective while maintaining reliable performance. 
This flexibility opens opportunities for widespread adoption, 
especially in developing regions or in contexts where large-scale infrastructure investments are not feasible.

Looking ahead, 
several enhancements are planned to increase the system's intelligence, 
responsiveness, 
and utility. 
These include integration with smart city frameworks to deliver real-time traffic and parking analytics, 
aiding not only drivers but also city planners and municipal authorities. 
Predictive modeling techniques using machine learning can be applied to forecast parking space availability based on historical usage patterns and live sensor inputs, 
optimizing traffic flow and minimizing time spent searching for parking. 
The architecture can also be extended into multi-camera or drone-assisted configurations, 
enabling coverage of vast and dynamic environments such as open-air lots or event venues. 
Drones on predetermined flight paths could provide aerial perspectives, 
increasing monitoring flexibility and responsiveness. 
By combining state-of-the-art machine vision with user-centric design principles, 
this system presents a forward-thinking, 
practical solution to the persistent challenges of urban parking, 
paving the way for smarter, more efficient cities.


\end{document}
