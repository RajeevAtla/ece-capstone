\documentclass{article}
\usepackage{geometry}
\geometry{margin=1in}
\usepackage{enumitem}
\usepackage{xcolor}

\begin{document}

\begin{center}
    \textcolor{red}{\textbf{\Large Rutgers ECE}} \\

    \vspace{1em}
    \textcolor{red}{\textbf{\Large Capstone Program Fall 2024 - Spring 2025}} \\

    \vspace{1em}
    \textcolor{red}{\textbf{\Large Project Scope of Work Info}} \\
\end{center}

\vspace{1em}
\noindent Please provide the following information to be shared with on capstone information exchange platform:

\vspace{2em}
\begin{enumerate}[leftmargin=1.5cm]
    \item \textbf{\large Project Number:} S25-02\\
    
    \vspace{1em}

    \item \textbf{\large  Project title (as will appear on the poster): Parking Assist Bot} \\
    
    \vspace{1em}

    \item \textbf{\large Team members: Rajeev Atla, Parshva Mehta, Aman Patel, Abhiram Vemuri} \\
    
    \vspace{1em}

    \item \textbf{\large Adviser(s) name(s): Kristin Dana} \\
    
    \vspace{1em}

    \item \textbf{\large Up to 5 keywords that will help to classify the project scope: Computer Vision, Machine Learning, Mobile App Development} \\
    
    \vspace{1em}

    \item \textbf{\large Project scope of work (up to 1000 words):} \\
    \textit{[General guidelines: The abstract should include a discussion about the problem addressed in the project; background review of the state of the art in the relevant field; objective of the proposed projects; and the adopted approach.]}
    
    Parking has become an issue for everyone who uses automobile transport, 
    especially in populated areas throughout the United States. 
    As a major center of student, 
    social, 
    and career life, 
    Rutgers University is emblematic of this problem. 
    Thousands of students, 
    faculty, 
    and staff spend their valuable time driving to campus and still grapple with the issue of where to park their cars daily. 
    Finding parking in a packed lot can consume fuel and waste time. 
    The problems with this extend to productivity and environmental conservation.

    The parking management challenge stems from issues like unusable parking space information. 
    Present strategies to address the parking problem, 
    like signage depicting the lot status, 
    smartphone applications with information entered manually, 
    or even physical lot greeters, 
    are inadequate or lack the real-time, 
    accurate information commuters can incorporate. 
    These inefficiencies are magnified at Rutgers University due to the size of the campus and the large commuter population.

    In this respect, 
    this project aims to develop a new robotic system that can move without intervention in on-campus parking lots, 
    detect empty parking spaces in real-time, 
    and inform users via a mobile application. 
    Therefore, 
    the idea is to revolutionize the parking experience meant for Rutgers commuters with the help of improvements in robotics, 
    computer vision, 
    and IoT applications. 
    The primary goal of this project is to mitigate time wastage and seek to add value to the parking system, 
    which might reduce unnecessary environmental effects.

    We can look at current parking management solutions to determine if this project is valuable. 
    Automatic parking systems available in intelligent cities include static sensors placed on the parking slot or lampposts to identify the state of the parking lot. 
    These systems can be effective in some contexts but involve significant investments in infrastructure and cannot be easily integrated into parking lots already in use. 
    Mobile applications, 
    in contrast, 
    base their information on user-provided data or, 
    in the case of the static parking lot capacity, 
    on updates that do not represent real-time situations in most cases. 
    However, 
    neither solution effectively handles the constantly changing and extensive parking needs of a dynamic, 
    crowded area such as Rutgers.

    Mobile and flexible robotic systems are being considered in numerous disciplines because of their autonomous capabilities and real-time data sampling. 
    Self-driven robots supplemented by sensors and intelligent algorithms are now available across agricultural and warehousing industries. 
    These applications can be applied to parking management solutions as well. 
    In this project, 
    we present a new solution that uses the benefits of robotic systems and state-of-the-art computer vision wireless communication to solve the problem of parking in a large university.

    The proposed solution stems from developing a mobile robot that can self-navigate within parking lots and provide the occupancy status of specific parking spaces. 
    The robot will also incorporate camera systems with high-definition resolution, 
    ultrasound systems, 
    and LIDAR systems to give correct estimations of empty and occupied spaces regardless of lighting or environmental conditions. 
    The data gathered by these sensors will be analyzed through machine learning algorithms to guarantee the detection of both vehicles and open spaces. 
    Real-time path planning means that the robot will traverse through parking lots along pre-designated paths but with room for changes in response to new conditions in the parking lots. 
    For example, some paths may be blocked while others have more cars than usual.

    Once parking data is recorded, 
    it will be relayed wirelessly to a network server as part of a mobile application that commuters can use. 
    It will display real-time parking updates on the map for users to scroll through, 
    with a user-friendly interface that allows users to filter the parking lots by nearby locations or preferred lots. 
    It will also offer updates when more data is collected. 
    This system will benefit not only the times taken individually in specific parking searches but also the overall function of parking resources within the campus.

    This project goes beyond the surface-level penetration, 
    as discussed in this paper. 
    From an individual perspective, 
    a commuter would experience improved travel time, 
    less stress, 
    and more certainty regarding parking. 
    On a high level, 
    this solution might help optimize Rutgers’ parking system operations, 
    thus possibly avoiding or postponing future expansions. 
    The environmental benefits are also apparent: 
    reducing the time drivers use their cars for cruising around in search of parking spaces also means that commuter vehicles will emit fewer emissions into the atmosphere, 
    which coincides with the university’s sustainability objectives.

\end{enumerate}

\end{document}
